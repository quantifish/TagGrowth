\documentclass[11pt, a4paper]{article}

\usepackage{amsfonts}
\usepackage{amssymb}
\usepackage{amsmath}
\usepackage{epsfig}
\usepackage{graphicx}
\usepackage{tabularx}
\usepackage{parskip}
\usepackage[margin=2.54cm]{geometry}
\usepackage[colorlinks=true]{hyperref}
\usepackage[sort]{natbib}

%%%%%%%%%%%%%%%%%%%%%%%%%%%%%%%%%%%%%%%%%%%%%%%%%%%%%%%%%%%%%%%%%%%%%%%%%%%%%%%%%
%%%%%%%%%%%%%%%%%%%%%%%%%%%%%%%%%%%%%%%%%%%%%%%%%%%%%%%%%%%%%%%%%%%%%%%%%%%%%%%%%
\title{Tag growth}

\author{D'Arcy N. Webber, Jim Thorson}

\date{\today}

\begin{document}
\maketitle

%%%%%%%%%%%%%%%%%%%%%%%%%%%%%%%%%%%%%%%%%%%%%%%%%%%%%%%%%%%%%%%%%%%%%%%%%%%%%%%%%
%%%%%%%%%%%%%%%%%%%%%%%%%%%%%%%%%%%%%%%%%%%%%%%%%%%%%%%%%%%%%%%%%%%%%%%%%%%%%%%%%
%%%%%%%%%%%%%%%%%%%%%%%%%%%%%%%%%%%%%%%%%%%%%%%%%%%%%%%%%%%%%%%%%%%%%%%%%%%%%%%%%
\section{Introduction}
%%%%%%%%%%%%%%%%%%%%%%%%%%%%%%%%%%%%%%%%%%%%%%%%%%%%%%%%%%%%%%%%%%%%%%%%%%%%%%%%%
%%%%%%%%%%%%%%%%%%%%%%%%%%%%%%%%%%%%%%%%%%%%%%%%%%%%%%%%%%%%%%%%%%%%%%%%%%%%%%%%%
%%%%%%%%%%%%%%%%%%%%%%%%%%%%%%%%%%%%%%%%%%%%%%%%%%%%%%%%%%%%%%%%%%%%%%%%%%%%%%%%%
\textit{Dissostichus mawsoni}

Key words: Antarctic toothfish, Ross Sea


%%%%%%%%%%%%%%%%%%%%%%%%%%%%%%%%%%%%%%%%%%%%%%%%%%%%%%%%%%%%%%%%%%%%%%%%%%%%%%%%%
%%%%%%%%%%%%%%%%%%%%%%%%%%%%%%%%%%%%%%%%%%%%%%%%%%%%%%%%%%%%%%%%%%%%%%%%%%%%%%%%%
%%%%%%%%%%%%%%%%%%%%%%%%%%%%%%%%%%%%%%%%%%%%%%%%%%%%%%%%%%%%%%%%%%%%%%%%%%%%%%%%%
\section{Simulation}
%%%%%%%%%%%%%%%%%%%%%%%%%%%%%%%%%%%%%%%%%%%%%%%%%%%%%%%%%%%%%%%%%%%%%%%%%%%%%%%%%
%%%%%%%%%%%%%%%%%%%%%%%%%%%%%%%%%%%%%%%%%%%%%%%%%%%%%%%%%%%%%%%%%%%%%%%%%%%%%%%%%
%%%%%%%%%%%%%%%%%%%%%%%%%%%%%%%%%%%%%%%%%%%%%%%%%%%%%%%%%%%%%%%%%%%%%%%%%%%%%%%%%
Simulated 315 individuals, the same number as in the actual toothfish data
set. The next step was to simulate sex, Age1, Age2 and time at liberty. What I
did in the current simulation run was:
\begin{itemize}
\item Sampled sex from the observed sexes of individuals (with replacement).
\item Sampled Age1, Age2 and time at liberty independently (with replacement)
  from those observed. Randomly selected one of these variables and calculated
  this values given the other two.
\item Rounded Age1, Age2 and time at liberty off to the nearest integer.
\end{itemize}
This is a bit of a hack I know, but the alternative did not yeild realistic
looking Age1, Age2, liberty samples.



\newpage\clearpage
%%%%%%%%%%%%%%%%%%%%%%%%%%%%%%%%%%%%%%%%%%%%%%%%%%%%%%%%%%%%%%%%%%%%%%%%%%%%%%%%%
%%%%%%%%%%%%%%%%%%%%%%%%%%%%%%%%%%%%%%%%%%%%%%%%%%%%%%%%%%%%%%%%%%%%%%%%%%%%%%%%%
\subsection{Fits to simulated data}
%%%%%%%%%%%%%%%%%%%%%%%%%%%%%%%%%%%%%%%%%%%%%%%%%%%%%%%%%%%%%%%%%%%%%%%%%%%%%%%%%
%%%%%%%%%%%%%%%%%%%%%%%%%%%%%%%%%%%%%%%%%%%%%%%%%%%%%%%%%%%%%%%%%%%%%%%%%%%%%%%%%

%%%%%%%%%%%%%%%%%%%%%%%%%%%%%%%%%%%%%%%%%%%%%%%%%%%%%%%%%%%%%%%%%%%%%%%%%%%%%%%%%
\subsubsection{sims1}
%%%%%%%%%%%%%%%%%%%%%%%%%%%%%%%%%%%%%%%%%%%%%%%%%%%%%%%%%%%%%%%%%%%%%%%%%%%%%%%%%
This was our first attempt at estimating simulated data.  Here $b$ was treated
as a random effect.  $L_0$ was estimated without a prior penalty.  Looking at
the histograms below there seems to be two states that the model estimates are
jumping between.  Looking at the trace plots, when $\sigma_o$ is estimated high
then the remaining parameters are poorly estimated.
\begin{table}[!htbp]
  \begin{quote}
  \caption{\label{tab:} .} \small{
  \begin{center}
  \begin{tabular}{lrr}
  \hline
  Parameter      & Female & Male\\
  \hline
  $L_0$          & 43   & 50\\
  $\overline{b}$ & 0.001 & 0.00106\\
  $\sigma_b$     & 0.21 & 0.19\\
  $\gamma$       & 0.19   & 0.19\\
  $\psi$         & 2e-10 & 2e-10\\
  $\sigma_o$     & 0.083 & 0.083\\
  $\sigma_z$     & 0.0   & 0.0\\
  $\sigma_y$     & 0.0   & 0.0\\
  \hline
  \end{tabular}
  \end{center}
  }
  \end{quote}
\end{table}

\begin{figure}[!htbp]
  \centering
  \includegraphics[width=\linewidth]{../simulation/sims1/results/SimPars.png}
  \includegraphics[width=\linewidth]{../simulation/sims1/results/TracePars.png}
  \begin{quote}
    \caption{pdH fits plotted only (80 of 100 fits were pdH)..}
  \label{fig:}
  \end{quote}
\end{figure}


\newpage\clearpage
%%%%%%%%%%%%%%%%%%%%%%%%%%%%%%%%%%%%%%%%%%%%%%%%%%%%%%%%%%%%%%%%%%%%%%%%%%%%%%%%%
\subsubsection{sims2}
%%%%%%%%%%%%%%%%%%%%%%%%%%%%%%%%%%%%%%%%%%%%%%%%%%%%%%%%%%%%%%%%%%%%%%%%%%%%%%%%%
In this simulation/estimation we placed a prior penalty on $L_0$ (a normal
centered about 0).  The simulation was run with $L_0$ values much closer to 0.
Again, $b$ was treated as a random effect.  Doing a poor job of recovering
$L_0$ for males.  Two states that seem to be linked to $\sigma_o$ again.  Not
getting $\psi$ at all.
\begin{table}[!htbp]
  \begin{quote}
  \caption{\label{tab:} .} \small{
  \begin{center}
  \begin{tabular}{lrr}
  \hline
  Parameter      & Female & Male\\
  \hline
  $L_0$          & 0.0   & 6.9\\
  $\overline{b}$ & 0.003 & 0.003\\
  $\sigma_b$     & 0.106 & 0.112\\
  $\gamma$       & 0.4   & 0.4\\
  $\psi$         & 0.001 & 0.001\\
  $\sigma_o$     & 0.099 & 0.099\\
  $\sigma_z$     & 0.0   & 0.0\\
  $\sigma_y$     & 0.0   & 0.0\\
  \hline
  \end{tabular}
  \end{center}
  }
  \end{quote}
\end{table}

\begin{figure}[!htbp]
  \centering
  \includegraphics[width=\linewidth]{../simulation/sims2/results/SimPars.png}
  \includegraphics[width=\linewidth]{../simulation/sims2/results/TracePars.png}
  \begin{quote}
    \caption{pdH fits plotted only (57 of 100 fits were pdH).}
  \label{fig:}
  \end{quote}
\end{figure}

\begin{figure}[!htbp]
  \centering
  \includegraphics[width=\linewidth]{../simulation/sims2/results/IndivGrowth_3.png}
  \includegraphics[width=\linewidth]{../simulation/sims2/results/ObsVsPred_3.png}
  \begin{quote}
    \caption{The first pdH simulation (sim 3).}
  \label{fig:}
  \end{quote}
\end{figure}



%%%%%%%%%%%%%%%%%%%%%%%%%%%%%%%%%%%%%%%%%%%%%%%%%%%%%%%%%%%%%%%%%%%%%%%%%%%%%%%%%

%\bibliographystyle{agsm}
%\bibliography{refs/myrefs}

%%%%%%%%%%%%%%%%%%%%%%%%%%%%%%%%%%%%%%%%%%%%%%%%%%%%%%%%%%%%%%%%%%%%%%%%%%%%%%%%%

\end{document}




I also tried this:
\begin{itemize}
\item Sampled sex using a binomial distribution.
\item Fit lognormal distributions to Age1 and time at liberty and simulate
  independently from the distributions.
\item Calculate Age2 given Age1 and time at liberty.
\item Rounded Age1, Age2 and time at liberty off to the nearest integer.
\end{itemize}
But this resulted in unrealistic Age2's (i.e. when a long time at liberty is
added to an already old fish), the plots just looked a bit silly.

Below are some exmaple plots using the first sampling approach above then
running these through our simulation model. 
\begin{figure}[!htbp]
  \centering
  \includegraphics[width=0.49\linewidth]{../simulation/sims/growth-1.png}
  \includegraphics[width=0.49\linewidth]{../simulation/sims/growth-2.png}
  \includegraphics[width=0.49\linewidth]{../simulation/sims/growth-6.png}
  \includegraphics[width=0.49\linewidth]{../simulation/sims/growth-45.png}
  \includegraphics[width=0.49\linewidth]{../simulation/sims/growth-20.png}
  \includegraphics[width=0.49\linewidth]{../simulation/sims/growth-29.png}
  \begin{quote}
    \caption{Data sets for two models that were not pdH [top, simulations 1 and
      2], failed to converge [middle, simulations 6 and 45] and the only two
      datasets that were pdH [bottom, simulations 20 and 29].}
  \label{fig:1}
  \end{quote}
\end{figure}




\newpage\clearpage
%%%%%%%%%%%%%%%%%%%%%%%%%%%%%%%%%%%%%%%%%%%%%%%%%%%%%%%%%%%%%%%%%%%%%%%%%%%%%%%%%
\subsubsection{No random effects}
%%%%%%%%%%%%%%%%%%%%%%%%%%%%%%%%%%%%%%%%%%%%%%%%%%%%%%%%%%%%%%%%%%%%%%%%%%%%%%%%%
Here I tried turning of all random effects.
\begin{table}[!htbp]
  \begin{quote}
  \caption{\label{tab:} .} \small{
  \begin{center}
  \begin{tabular}{lrr}
  \hline
  Parameter      & Female & Male\\
  \hline
  $L_0$          & 0.0   & 6.9\\
  $\overline{b}$ & 0.003 & 0.003\\
  $\sigma_b$     & 0.0   & 0.0\\
  $\gamma$       & 0.4   & 0.4\\
  $\psi$         & 0.001 & 0.001\\
  $\sigma_o$     & 0.099 & 0.099\\
  $\sigma_z$     & 0.0   & 0.0\\
  $\sigma_y$     & 0.0   & 0.0\\
  \hline
  \end{tabular}
  \end{center}
  }
  \end{quote}
\end{table}

\begin{figure}[!htbp]
  \centering
  \includegraphics[width=\linewidth]{../simulation/sims/results/SimPars.png}
  \includegraphics[width=\linewidth]{../simulation/sims/results/TracePars.png}
  \begin{quote}
    \caption{.}
  \label{fig:}
  \end{quote}
\end{figure}
